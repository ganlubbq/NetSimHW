\documentclass[10pt]{article}

\usepackage[applemac]{inputenc}
\usepackage[english]{babel}
\usepackage[T1]{fontenc}
\usepackage{cite, url,color} % Citation numbers being automatically sorted and properly "compressed/ranged".
%\usepackage{pgfplots}
\usepackage{graphics,amsfonts}
\usepackage[pdftex]{graphicx}
\usepackage[cmex10]{amsmath}
% Also, note that the amsmath package sets \interdisplaylinepenalty to 10000
% thus preventing page breaks from occurring within multiline equations. Use:
 \interdisplaylinepenalty=2500
% after loading amsmath to restore such page breaks as IEEEtran.cls normally does.

%% Useful packages for creation of two-column and more complex figures
% Compact lists
\usepackage{enumitem}
\usepackage{booktabs}
\usepackage{fancyvrb}

\usepackage{listings} % inserisce listati di programmi
\definecolor{commenti}{rgb}{0.13,0.55,0.13}
\definecolor{stringhe}{rgb}{0.63,0.125,0.94}
\lstloadlanguages{Matlab}
\lstset{% general command to set parameter(s)
framexleftmargin=0mm,
frame=single,
keywordstyle = \color{blue},% blue keywords
identifierstyle =, % nothing happens
commentstyle = \color{commenti}, % comments
stringstyle = \ttfamily \color{stringhe}, % typewriter type for strings
showstringspaces = false, % no special string spaces
emph = {for, if, then, else, end},
emphstyle = \color{blue},
firstnumber = 1, % numero della prima linea
numbers =right, %  show number_line
numberstyle = \tiny, % style of number_line
stepnumber = 5, % one number_line after stepnumber
numbersep = 5pt,
language = {Matlab}, % per riconoscere la sintassi matlab
extendedchars = true, % per abilitare caratteri particolari
breaklines = true, % per mandare a capo le righe troppo lunghe
breakautoindent = true, % indenta le righe spezzate
breakindent = 30pt, % indenta le righe di 30pt
basicstyle=\footnotesize\ttfamily
}

%Pseudocode package
\usepackage{algorithm}
\usepackage[noend]{algpseudocode}

\usepackage{array}
% http://www.ctan.org/tex-archive/macros/latex/required/tools/
\usepackage{mdwmath}
\usepackage{mdwtab}
%mdwtab.sty	-- A complete ground-up rewrite of LaTeX's `tabular' and  `array' environments.  Has lots of advantages over
%		   the standard version, and over the version in `array.sty'.
% *** SUBFIGURE PACKAGES ***
\usepackage[tight,footnotesize]{subfigure}

\usepackage[top=2cm, bottom=2cm, right=1.6cm,left=1.6cm]{geometry}
\usepackage{indentfirst}

%\usepackage{times}
%\usepackage[active]{srcltx}

\setlength\parindent{0pt}
\linespread{1}

\def\C#1{\mathcal{#1}}

\usepackage{mathtools}
\DeclarePairedDelimiter{\ceil}{\lceil}{\rceil}
\DeclarePairedDelimiter{\floor}{\lfloor}{\rfloor}
\DeclareMathOperator*{\argmin}{arg\,min}
\DeclareMathOperator*{\argmax}{arg\,max}


% Package used to keep inherent figures in the same section
\usepackage{placeins}


\begin{document}
\title{Network Analysis and Simulation - Homework 3}
\author{Michele Polese, 1100877}

\maketitle

\section*{Exercise 1 - Queue simulations}
In this exercise I simulated the behavior of two different queues. They are both slotted, the first one has deterministic service, in one slot, and arrivals in each slot distributed in the following way
\begin{equation}
  \mbox{number of arrivals} = 
  \begin{cases}
    2 & \mbox{with probability} a\\
    1 & \mbox{with probability} a\\
    0 & \mbox{with probability} 1-2a
  \end{cases}
  \label{eq:queue1arr}
\end{equation}
Therefore the average rate of arrivals is $\lambda_1 = 3a$, the service rate $\mu_1 = 1$ and $\rho = \frac{\lambda_1}{\mu_1} = 3a$. 
The goal of the simulation is to study as first the throughput-delay tradeoff, in particular for $a \in [0, 1/3]$ study the delay of the queue as a function of $\rho$. Note that for $a = 1/3$ then $\rho = 1$, therefore the queue is unstable. 
%% TODO check this sentence 
The queue behaves as follows. At the beginning of each slot, if there's at least one user, the first user who arrived is served. Then the arrivals take place, with the probabilities of~\eqref{eq:queue1arr}, and finally the number of users in the queue is sampled. Since the service process is deterministic of $N_s$ solts, there's a counter that in presence of users in the queue increases at each slot of one unit until it reaches the given $N_s$. In particular the users cannot be served in the same slot in which they arrive. The slot in which they begin their service time already counts as 1, therefore the counter of the remaining slots is initialized to 1 and not to 0. Since in this queue the service time $N_s = 1$, then the first slot in which a user enters in service is also the slot in which it leaves the queue. However the MATLAB function that simulates the queue is general and it can handle any deterministic service time. 

In this implementation the delay of each departed packet is measured with the help of a queue. In MATLAB this is an array whose last position is filled with the couple $(slot, n)$ at each arrival, where $slot$ is the index of the time slot in which there is the $n$-th arrival. If there are two arrival, then they are both inserted in the queue with the same $slot$ value and an increasing $n$. When a there's a departure, instead, the first position of the queue is read, and the difference between current slot index and the $slot$ value is stored in an array. Then the first couple of the queue is popped out and every element shifts by one. 

The function that simulates the queue finally returns an array with the number of users for each slot and a second one with the delay for each departed user, in order to let the main class handle statistics. In particular, for each value of $a$ (which is sampled in $[0.1, 0.9]/3$), I repeat 100 indipendent simulations, each of length $n_{slot} = 10^6$ slots. For each simulation, given the number $n_{d}$ of departed users, I compute the mean delay as 
\begin{equation}
  dl_{sim} = \frac{1}{n_d} \sum_{i = 0}^{n_d} dl_{user_i}
\end{equation}
and then I average over the 100 simulations to get a lower variance. The results are in Figure

I also studied the behavior of one realization of the queue in $10^4$ slots. The results are in Figure

The second queue, instead, has a bernoulli arrival process, with one arrival with probability $1/2$ in each slot, and a geometric service time of average $1/b$. Since the arrival rate is $\lambda_2 = 1/2$ and the service rate is $\mu_2 = b$, then the utilization factor is $\rho = \frac{1}{2b}$. Also for this queue I studied the throughput-delay trade-off, with the events happening in each slot as in the previous queue and with delay measured for each departed packet with a FIFO queue. Note that since departures are geometric, they are memoryless and the counter of the number of slots spent in service is useless. Therefore at every slot, if there's at least one user, it departs with probability $b$ and it remains in service with probability $1-b$. 

The results 

The realizations are

The second element that I studied on each of this queues is the probability of overflow with a fixed sized queue, for stable $\rho$. The probability of overflow is given by 
\begin{equation}
  P_{of} = \frac{\mbox{dropped packets}}{\mbox{dropped packets} + \mbox{served packets}}
\end{equation}
where the denominator is equal to the total number of arrivals in the system. 

I studied this probability with 2 dedicated functions for each type of queue, which handle the departures and arrivals as before but don't track the number of users in the system and the delay metric in order to speed up the simulation. The length of each simulation is indeed a critical issue. In order to estimate a very low probability very long simulations must be run, because otherwise the number of obeserved events would be to small to provide a reliable estimate. In particular, for each simulation with $n_{arr}$ arrivals, a general result from \cite{leb} states that if the number of drop events observed is $z \ge 6$ and $n_{arr} - z \ge 6$ than it is possible to apply a normal approximation for the confidence interval at a certain level $\gamma$, i.e.
\begin{equation}
  \begin{cases}
  L(z) \approx \frac{z}{n} - \frac{\eta}{n}\sqrt{z\left(1 - \frac{z}{n}\right)} \\
  U(z) \approx \frac{z}{n} + \frac{\eta}{n}\sqrt{z\left(1 - \frac{z}{n}\right)}
  \end{cases}
\end{equation}
with $\eta$ such that $N_{0,1}(\eta) = \frac{1+\gamma}{2}$.

Since the target to reach is $P_{of} \le 10^-5$, then the chosen length of the simulation in number of slots will be of $n_{slots} = \frac{100}{\lambda_i P_{of}}$ in order to observe on average at least $\frac{100}{\lambda_i}$ events. 

The experiment is repeated $N$ times, in order to reduce the variance, and the final probability of overflow for a given size is computed as the sample mean of the $N$ probability of overflow. The size chosen is the one that guarantees that the upper bound of the confidence interval for the sample mean estimator is below the required threshold of $10^{-5}$. 

In Figure 

In Figure, instead... 

\section*{Cellular system and Aloha}
In a slotted Aloha system it is interesting to study which is the probability that some transmission are successful even in presence of interferers. 
Each user in the system which has to communicate with a central base station is characterized by the following quantities:
\begin{itemize}
\item distance $r$ from the BS. Since the users are uniform in a circle of radius 1, then the probability density function of this distribution is $h(r) = 2r$.
\item $\eta$ power loss law exponent, which in the following simulations is set to 4
\item a log-normal random variable $e^{\xi}$ that describes shadowing, with $\xi ~ N_{0,\sigma^2}$. Note that the given $\sigma_{dB}= 8$ is in dB, thus $\sigma = (0.1 \log_e(10))\sigma_{dB}$
\item a Rayleigh distributed r.v. with unit power $\sigma_R = 1$. Note that a Rayleigh r.v. can be generated as $\sigma_R \sqrt{-2*\log_e(X)}$ with $X \sim N_{0,1}$.
\end{itemize}
The received power from a transmitter at distance $r$ is then $P_R = R^2 e^{\xi} K r^{-\eta} P_T$. We assume that a constant term $K P_T$ and the exponent $\eta$ are equal for all the users, and also that the Rayleigh fading and the log-normal shadowing are iid. 

Then the signal to interference ratio of a user against $k$ interferers can be expressed as
\begin{equation}
  SIR = \frac{P_{R0}}{\sum_{i=1}^{k} P_{Ri}} = \frac{R_0^2 e^{\xi_0}}{\sum_{i = 1}^{k}R_i^2 e^{\xi_i}\left(\frac{r_0}{r}\right)^{\eta}}
\end{equation}
There is a successful transmission for this user with probability 
\begin{equation}
  P_s = P[SIR > b]
\end{equation}
with b a suitable threshold, that in the following simulations will be set to 6 and 10 dB. 

This quantity can be computed analytically as it is done in \cite{capture} and the final result of the probability of having a success for a user at distance $r_0$ given that $n$ packets are transmitted (thus $n-1$ interferers) is
\begin{equation}
  P_n(r_0) = \int_{-\infty}^{\infty} \frac{d\xi_0}{\sqrt{2\pi}\sigma} e^{-\frac{\xi_0^2}{2\sigma^2}}[I(\xi_0, r_0)]^{n-1}
\end{equation}
with 
\begin{equation}
  I(\xi_0, r_0) = \int_{-\infty}^{\infty} \frac{d\xi}{\sqrt{2\pi}\sigma}e^{-\frac{\xi^2}{2\sigma^2}} \int_{0}^{1} \frac{h(r) dr}{1+be^{\xi - \xi_0}\left(\frac{r}{r_0}\right)^{-\eta}}
\end{equation}
Then the capture probability, that describes the probability that the strongest transmission can be successfully received even with $n-1$ interferers, is
\begin{equation}
  C_n = \int_0^1 nP_n(r_0)h(r_0) dr_0 = \int_{-\infty}^{\infty} \frac{d\xi_0}{\sqrt{2\pi}\sigma} e^{-\frac{\xi_0^2}{2\sigma^2}} \int_0^1 n h(r_0) [I(\xi_0, r_0)]^{n-1} dr_0
\end{equation}

In this homework I evaluated these probabilities using both a Montecarlo simulation and numerical integration, for a total number of users up to 30. The Montecarlo simulation computes directly the SIR and compares it to the desired threshold. In particular, in order to speed up the simulation, at each iteration the simulator generates the complete set of random variables as described previously for all the $N_{tot} = 30$ users. Then it computes the SIR as if there were 2 users, 3 users and so on, sampling from the $N_{tot}$ random variables, and compares it with the threshold $b$. It uses a matrix $\mathbf{P}$ of size $n_{sim} \times N_{tot}$ in order to store at each iteration of the simulation $i$ the value
\begin{equation}
P(i, n) = 
\begin{cases}
  n & \mbox{if } SIR > b \mbox{ with } n \mbox{ packets transmitted}\\
  0 & \mbox{otherwise}
\end{cases}
 \quad n = 2, \dots, N_{tot}
\end{equation}
while $P(i, 1) = 1$ since if there is only one transmission it is successful (in this analysis the noise is not taken into account).
Then each for each column $n$ of the matrix the estimate of $C_n$ is computed as
\begin{equation}
  \hat{C}_n = \frac{1}{n_{sim}} \sum_{i = 0}^{n_{sim} - 1} P(i, n)
\end{equation}

\section*{GeRaF}


\begin{thebibliography}{10}

\bibitem{leb}
Y. Le Boudec, Performance Evaluation of Computer and Communications Systems, EPFL, 2015

\bibitem{capture}
M. Zorzi and R. R. Rao, Capture and retransmission control in
mobile radio, IEEE J. Sel. Areas Commun., vol. SAC-12, no. 8, pp.
1289 - 1298, Oct. 1994


\end{thebibliography}

\end{document}
